\usepackage{url} % make URLs actually work
\usepackage[hidelinks]{hyperref}
\usepackage{courier} % use courier font for texttt env
\usepackage{graphicx, subcaption} % pictures
\usepackage{cutwin} % Text wrapping in theorem environments
\usepackage{ifthen}

\usepackage{fmtcount} % String of number

\usepackage{amsmath, amssymb, amsthm, amsfonts, stmaryrd, mathtools, dsfont}
\usepackage[most]{tcolorbox}
\tcbuselibrary{many}
\usepackage{varwidth}
\usepackage[margin=1cm]{caption} % Have a nice 1cm margin round captions so they don't go right to the edge

\usepackage{nicefrac} % Nice fractions in text

\usepackage{booktabs}

\usepackage{textcomp} % Prevents "Not defining \perthousand" warning
\usepackage{gensymb}

\usepackage{venndiagram}

\usepackage{bibentry}
\nobibliography*

\usepackage[titletoc]{appendix}

\newcommand\mathplus{+}

% Phantom chapters
\newcommand{\phantomchapter}[1]{%
  \chapter*{#1}%
  \phantomsection%
  \addcontentsline{toc}{chapter}{#1}%
  \vspace{1em}
}

% repeat caption
\newcommand{\repeatcaption}[1]{%
  \renewcommand{\thefigure}{\ref{#1}}%
  \captionsetup{list=no}%
  \caption{\nameref{#1}}%
  \addtocounter{figure}{-1}%
}

\newcommand{\repeatchapter}[1]{%
  \autoref{#1}: \nameref{#1}%
}

\usepackage{etoolbox}
% Patch \footnotemark to provide a label that can be referenced.
\patchcmd{\footnotemark}{\stepcounter{footnote}}{\refstepcounter{footnote}}{}{}

% ================================= blank page =================================
\usepackage{afterpage}
\newcommand\blankpage{%
  \null
  \thispagestyle{empty}%
  \newpage}

% ================================= todonotes ==================================
% Make todo notes capable of going in floats
\setlength{\marginparwidth }{2cm}
\usepackage{todonotes}
\usepackage{marginnote}
\renewcommand{\marginpar}{\marginnote}

% ============================= Bibliography stuff =============================
\usepackage[nottoc,notlof,notlot]{tocbibind}
\renewcommand\bibname{References}

% ======================== Chapter and section headings ========================
\usepackage{titlesec}
\def\Vhrulefill{\leavevmode\leaders\hrule height 0.7ex depth \dimexpr0.4pt-0.7ex\hfill\kern0pt}

\titleformat{\chapter} % command
[display] % shape
{\bfseries\large} % format
{\vspace{-3cm}\centering\Vhrulefill~\chaptertitlename \ \thechapter~\Vhrulefill} % label
{0ex} % sep
{\vspace{1ex}\centering\Large} % before-code
[
  \vspace{-0.5ex}%
  \rule{\textwidth}{0.3pt}
  \vspace{-1.5cm}
] % after-code

% Keep section numberings
% \usepackage{etoolbox}
% \makeatletter
% \patchcmd{\ttlh@hang}{\parindent\z@}{\parindent\z@\leavevmode}{}{}
% \patchcmd{\ttlh@hang}{\noindent}{}{}{}
% \makeatother

% Contributions
\usepackage{chngcntr}

\titleclass{\contribution}{straight}[\section]
\newcounter{contribution}
\titleformat{\contribution}[runin]{\normalfont\bfseries}{}{0pt}{C\thecontribution: }
\titlespacing*{\contribution}{0pt}{0pt}{3pt}

\makeatletter
\newcommand{\l@contribution}{\@dottedtocline{5}{1.5em}{2.3em}}
\renewcommand{\l@contribution}{\@dottedtocline{5}{1.5em}{2.3em}}
\providecommand*{\toclevel@contribution}{5}%     %% <---
\makeatother

\usepackage[linewidth=0.5pt]{mdframed}

\setcounter{tocdepth}{1}
% \newcommand{\contrib}[2]{\refstepcounter{contribution}\paragraph{C\arabic{contribution}: #1 \label{#2}}}
\newcommand{\repeatcontrib}[1]{%
  % \vspace{0.5em}\noindent\fbox{\parbox{\columnwidth}{C\ref{#1}: \nameref{#1}}}%
  \begin{mdframed}
    C\ref{#1}: \nameref{#1}.
  \end{mdframed}
}
\setcounter{secnumdepth}{3} % default value for 'report' class is "2"


% ============================ Contents page stuff =============================
\usepackage{tocloft}
\providecommand{\cftchapafterpnum}{\vspace{10pt}}
\setlength{\cftbeforetoctitleskip}{-3em}
\setlength{\cftbeforechapskip}{6pt}


% ============================ Header and footer ==============================
\usepackage{emptypage} % Remove header/footer on empty pages
\makeatletter
%preamble
\if@twoside%
  \usepackage{fancyhdr}
  \setlength{\headheight}{25pt} % TODO: See if you can remove this

  \fancypagestyle{main}{%
    \fancyhf{}
    \fancyhead[LE]{\rightmark}
    \fancyhead[RO]{\leftmark}
    \fancyfoot[C]{\thepage}
  }

  \addtocontents{toc}{\protect\thispagestyle{empty}}

\else%
  %%% put the stuff for false here (twoside=false)
  \pagestyle{plain}
\fi%
\makeatother

\usepackage{enumitem} % Mess about with itemize, enumerate, description styles

% ================================= algorithms =================================
\usepackage[noend]{algpseudocode}
\usepackage{algorithm}

\algrenewcommand{\alglinenumber}[1]{\color{gray}\footnotesize#1:}

% New definitions
\algnewcommand\algorithmicswitch{\textbf{switch}}
\algnewcommand\algorithmiccase{\textbf{case}}
% \algnewcommand\algorithmicdo{\textbf{do}}
\algnewcommand\Assert[1]{\State \algorithmicassert(#1)}%
% New "environments"
\algdef{SE}[SWITCH]{Switch}{EndSwitch}[1]{\algorithmicswitch\ #1\ \algorithmicdo}{\algorithmicend\ \algorithmicswitch}%
\algdef{SE}[CASE]{Case}{EndCase}[1]{\algorithmiccase\ #1}{\algorithmicend\ \algorithmiccase}%
\algdef{SE}[DO]{Do}{EndCase}[1]{\algorithmicdo\ #1}{\algorithmicend\ \algorithmicdo}%

\algtext*{EndSwitch}%
\algtext*{EndCase}%
\algtext*{DoWhile}%

\makeatletter
\newcounter{algorithmicH}% New algorithmic-like hyperref counter
\let\oldalgorithmic\algorithmic
\renewcommand{\algorithmic}{%
  \stepcounter{algorithmicH}% Step counter
  \oldalgorithmic}% Do what was always done with algorithmic environment
\renewcommand{\theHALG@line}{ALG@line.\thealgorithmicH.\arabic{ALG@line}}
\providecommand*{\toclevel@algorithm}{1}
\makeatother

% ================================= theorems ===================================
\newtheorem{theorem}{Theorem}[section]
\newtheorem{corollary}{Corollary}[theorem]
\newtheorem{lemma}[theorem]{Lemma}
{\theoremstyle{definition} \newtheorem{example}{Example}[section]}
{\theoremstyle{definition} \newtheorem{definition}{Definition}}

\tcolorboxenvironment{definition}{%
  enhanced jigsaw,
  sharp corners,
  colframe=black,
  interior hidden,
  parbox=false,
  boxrule=0.5pt,
  left=1mm,
  right=1mm,
  top=1mm,
  bottom=1mm
}

\tcolorboxenvironment{example}{%
  blanker,
  parbox=false,
  breakable,
  enlarge top by=0.5em,
  left=3mm,
  pad at break=0px,
  borderline west={0.5mm}{0pt}{black, -}
}

% ================================= autoref ====================================
\def\Appendixautorefname{Appendix}
\def\chapterautorefname{Chapter}
\def\sectionautorefname{Section}
\def\subsectionautorefname{Subsection}
\newcommand{\exampleautorefname}{Example}
\newcommand{\definitionautorefname}{Definition}
\newcommand{\algautorefname}{Algorithm}
\newcommand{\algorithmautorefname}{Algorithm}

\makeatletter
\patchcmd{\ALG@step}{\addtocounter{ALG@line}{1}}{\refstepcounter{ALG@line}}{}{}
\newcommand{\ALG@lineautorefname}{Line}
\makeatother


\newlist{questions}{enumerate}{2}
\setlist[questions]{label=RQ\arabic*, font=\bfseries, ref=\arabic*, wide=0pt, leftmargin=*}
\newcommand{\RQ}[1]{\hyperref[#1]{RQ\ref*{#1}}}


% ================================= listings ===================================
\usepackage{listings}
\lstset{%
  basicstyle=\ttfamily,
  breaklines=true,
  numbers=left,
  xleftmargin=2em,
  frame=none,
  framexleftmargin=2ex,
  stepnumber=1,
  numberstyle={\color{gray}\ttfamily}
}

\definecolor{solarised-background}{HTML}{9facad}
\definecolor{solarised-text}{HTML}{000203}

\newcommand{\commandshell}[1]{%
  \centering
  \begin{tikzpicture}[node distance=0pt]
    \node[shape=rectangle, fill=black] (top) {\parbox{0.9\textwidth}{\hfill\\}};
    \node[shape=rectangle, fill=black, rounded corners=5, shift={(0, 5pt)}] {\parbox{0.9\textwidth}{\hfill\\}};
    \node[shape=rectangle, fill=black, above = 1pt of top.south, anchor=north] {\parbox{0.9\textwidth}{\hfill\\}};

    \node[shape=circle, fill=red, shift={(6.3cm, 0)}, inner sep = 3pt] {};
    \node[shape=circle, fill=orange, shift={(5.8cm, 0)}, inner sep = 3pt] {};
    \node[shape=circle, fill=green, shift={(5.3cm, 0)}, inner sep = 3pt] {};
    \node[shape=rectangle, fill=solarised-background, below = 5pt of top.south, anchor=north] {\color{solarised-text}\parbox{0.9\textwidth}{\texttt{#1}}};
  \end{tikzpicture}
}
